% %%%%%%%%%%%%%%%%%%%%%%%%%%%%%%%%%%%%%%%%%%%%%%%%%%%%%%%%%%%%%%%%%%%%%%% %
%                                                                         %
% The Project Gutenberg EBook of The Origin and Development of the Quantum%
% Theory, by Max Planck                                                   %
%                                                                         %
% This eBook is for the use of anyone anywhere at no cost and with        %
% almost no restrictions whatsoever.  You may copy it, give it away or    %
% re-use it under the terms of the Project Gutenberg License included     %
% with this eBook or online at www.gutenberg.org                          %
%                                                                         %
%                                                                         %
% Title: The Origin and Development of the Quantum Theory                 %
%                                                                         %
% Author: Max Planck                                                      %
%                                                                         %
% Translator: H. T. Clarke                                                %
%             L. Silberstein                                              %
%                                                                         %
% Release Date: September 7, 2010 [EBook #33663]                          %
%                                                                         %
% Language: English                                                       %
%                                                                         %
% Character set encoding: ASCII                                           %
%                                                                         %
% *** START OF THIS PROJECT GUTENBERG EBOOK QUANTUM THEORY ***            %
%                                                                         %
% %%%%%%%%%%%%%%%%%%%%%%%%%%%%%%%%%%%%%%%%%%%%%%%%%%%%%%%%%%%%%%%%%%%%%%% %

\def\ebook{33663}
%%%%%%%%%%%%%%%%%%%%%%%%%%%%%%%%%%%%%%%%%%%%%%%%%%%%%%%%%%%%%%%%%%%%%%
%%                                                                  %%
%% Packages and substitutions:                                      %%
%%                                                                  %%
%% book:     Required.                                              %%
%% lmodern:  Latin-1 modern font. Required.                         %%
%% fancyhdr: Enhanced running headers and footers. Required.        %%
%% verbatim: for PG headers only.                                   %%
%%                                                                  %%
%% Producer's Comments:                                             %%
%%                                                                  %%
%%   None                                                           %%
%%                                                                  %%
%% PDF pages: 23                                                    %%
%% PDF page size: Letter paper                                      %%
%% Images: none                                                     %%
%%                                                                  %%
%% Summary of log file:                                             %%
%% * No overfull hboxes.                                            %%
%% * Four underfull hboxes.                                         %%
%%                                                                  %%
%% Compile History:                                                 %%
%%                                                                  %%
%% September, 2010: rfrank (fadedpage)                              %%
%%                  texlive2009, Mac OS 10.6                        %%
%%                                                                  %%
%% Command block:                                                   %%
%%                                                                  %%
%%     pdflatex x2                                                  %%
%%                                                                  %%
%%                                                                  %%
%% September 2010: pglatex.                                         %%
%%   Compile this project with:                                     %%
%%   pdflatex 33663-t.tex ..... TWO times                           %%
%%                                                                  %%
%%   pdfTeXk, Version 3.141592-1.40.3 (Web2C 7.5.6)                 %%
%%                                                                  %%
%%%%%%%%%%%%%%%%%%%%%%%%%%%%%%%%%%%%%%%%%%%%%%%%%%%%%%%%%%%%%%%%%%%%%%

\listfiles

\documentclass[12pt,oneside]{book}
\usepackage[letterpaper,top=1.0in,bottom=1.1in,left=1.1in,right=1.1in]{geometry}
\usepackage[T1]{fontenc}
\usepackage{lmodern}
\usepackage{fancyhdr}
\usepackage{verbatim}
\setlength{\headheight}{15.2pt}
\pagestyle{fancy}
\fancyhf{}
\renewcommand{\headrulewidth}{0pt}
\setlength{\parindent}{15pt}

\begin{document}

\pagenumbering{alph}

\begin{verbatim}
The Project Gutenberg EBook of The Origin and Development of the Quantum
Theory, by Max Planck

This eBook is for the use of anyone anywhere at no cost and with
almost no restrictions whatsoever.  You may copy it, give it away or
re-use it under the terms of the Project Gutenberg License included
with this eBook or online at www.gutenberg.org


Title: The Origin and Development of the Quantum Theory

Author: Max Planck

Translator: H. T. Clarke
            L. Silberstein

Release Date: September 7, 2010 [EBook #33663]

Language: English

Character set encoding: ASCII

*** START OF THIS PROJECT GUTENBERG EBOOK QUANTUM THEORY ***
\end{verbatim}

\mainmatter
\pagenumbering{arabic}

\begin{center}
\textbf{\Large THE ORIGIN AND DEVELOPMENT}\\
\vspace{8mm}
\textbf{OF THE}\\
\vspace{8mm}
\textbf{\Huge QUANTUM THEORY}\\
\vspace{10mm}
BY\\
\vspace{5mm}
{\Large MAX PLANCK}\\
\vspace{10mm}
TRANSLATED BY\\
\vspace{5mm}
\textbf{H.~T.~CLARKE} {\scriptsize AND} \textbf{L.~SILBERSTEIN}\\
\vspace{7mm}
\rule{30mm}{0.3mm}\\
\vspace{7mm}
{\small BEING THE}\\
\vspace{5mm}
\textbf{\large NOBEL PRIZE ADDRESS}\\
\vspace{5mm}
{\small DELIVERED BEFORE}\\
\vspace{5mm}
THE ROYAL SWEDISH ACADEMY OF SCIENCES\\
\vspace{5mm}
AT STOCKHOLM, 2 JUNE, 1920\\
\vspace{7mm}
\rule{30mm}{0.3mm}\\
\vspace{7mm}
\textbf{O$\,$X$\,$F$\,$O$\,$R$\,$D}\\
\vspace{4mm}
\textbf{AT THE CLARENDON PRESS}\\
\vspace{4mm}
\textbf{1922}
\end{center}

\clearpage
\begin{center}
\vspace*{4cm}
OXFORD UNIVERSITY PRESS\\
\vspace{2mm}
London~~~~Edinburgh~~~~Glasgow~~~~Copenhagen\\
New York~~~~Toronto~~~~Melbourne~~~~Cape Town\\
Bombay~~~~Calcutta~~~~Madras~~~~Shanghai\\
\vspace{2mm}
HUMPHREY MILFORD\\
Publisher to the University
\end{center}

\vfill
\begin{center}
\small
Produced by Roger Frank and the Online Distributed
Proofreading Team at http://www.fadedpage.net
\end{center}

\clearpage

\vspace*{3mm}
\begin{center}
\begin{Large}THE ORIGIN AND DEVELOPMENT OF\\
\vspace*{2mm}
THE QUANTUM THEORY\end{Large}
\end{center}
\vspace{5mm}
\chead{(\thepage)} % page numbering starts here
{\sc My} task this day is to present an address dealing with
the subjects of my publications. I feel I can best discharge
this duty, the significance of which is deeply
impressed upon me by my debt of gratitude to the
generous founder of this Institute, by attempting to sketch
in outline the history of the origin of the Quantum Theory
and to give a brief account of the development of this theory
and its influence on the Physics of the present day.

When I recall the days of twenty years ago, when the
conception of the physical quantum of `action' was first
beginning to disentangle itself from the surrounding mass
of available experimental facts, and when I look back upon
the long and tortuous road which finally led to its disclosure,
this development strikes me at times as a new illustration
of Goethe's saying, that `man errs, so long as he is striving.'
And all the mental effort of an assiduous investigator must
indeed appear vain and hopeless, if he does not occasionally
run across striking facts which form incontrovertible proof
of the truth he seeks, and show him that after all he has
moved at least one step nearer to his objective. The
pursuit of a goal, the brightness of which is undimmed by
initial failure, is an indispensable condition, though by no
means a guarantee, of final success.

In my own case such a goal has been for many years
the solution of the question of the distribution of energy in
the normal spectrum of radiant heat. The discovery by
Gustav Kirchhoff that the quality of the heat radiation
produced in an enclosure surrounded by any
emitting or absorbing bodies whatsoever, all at the same
temperature, is entirely independent of the nature of such
bodies(1)\footnote[1]{The numbers in brackets refer to the notes at the end of the article.},
established the existence of a universal function,
which depends only upon the temperature and the wave-length,
and is entirely independent of the particular properties
of the substance. And the discovery of this remarkable
function promised a deeper insight into the relation
between energy and temperature, which is the principal
problem of thermodynamics and therefore also of the
entire field of molecular physics. The only road to this
function was to search among all the different bodies
occurring in nature, to select one of which the emissive and
absorptive powers were known, and to calculate the energy
distribution in the heat radiation in equilibrium with that
body. This distribution should then, according to Kirchhoff's
law, be independent of the nature of the body.

A most suitable body for this purpose seemed H.~Hertz's
rectilinear oscillator (dipole) whose laws of emission for a
given frequency he had just then fully
developed(2). If a number of such oscillators be distributed in an enclosure
surrounded by reflecting walls, there would take place, in
analogy with sources and resonators in the case of sound,
an exchange of energy by means of the emission and
reception of electro-magnetic waves, and finally what is
known as black body radiation corresponding to Kirchhoff's
law should establish itself in the vacuum-enclosure. I expected,
in a way which certainly seems at the present day
somewhat na\"{i}ve, that the laws of classical electrodynamics
would suffice, if one adhered sufficiently to generalities and
avoided too special hypotheses, to account in the main for
the expected phenomena and thus lead to the desired goal.
I thus first developed in as general terms as possible the
laws of the emission and absorption of a linear resonator,
as a matter of fact by a rather circuitous route which might
have been avoided had I used the electron theory which
had just been put forward by H.~A.~Lorentz. But as I had
not yet complete confidence in that theory I preferred to
consider the energy radiating from and into a spherical
surface of a suitably large radius drawn around the
resonator. In this connexion we need to consider only
processes in an absolute vacuum, the knowledge of which,
however, is all that is required to draw the necessary conclusions
concerning the energy changes of the resonator.

The outcome of this long series of investigations, of
which some could be tested and were verified by comparison
with existing observations, e.~g. the measurements of
V.~Bjerknes(3) on damping, was the establishment of
a general relation between the energy of a resonator of
a definite free frequency and the energy radiation
of the corresponding spectral region in the surrounding
field in equilibrium with it(4). The remarkable result was
obtained that this relation is independent of the
nature of the resonator, and in particular of its coefficient
of damping---a result which was particularly welcome,
since it introduced the simplification that the energy of the
radiation could be replaced by the energy of the resonator,
so that a simple system of one degree of freedom could be
substituted for a complicated system having many degrees
of freedom.

But this result constituted only a preparatory advance
towards the attack on the main problem, which now
towered up in all its imposing height. The first attempt to
master it failed: for my original hope that the radiation
emitted by the resonator would differ in some characteristic
way from the absorbed radiation, and thus afford the
possibility of applying a differential equation, by the integration
of which a particular condition for the composition of
the stationary radiation could be reached, was not realized.
The resonator reacted only to those rays which were emitted
by itself, and exhibited no trace of resonance to neighbouring
spectral regions.

Moreover, my suggestion that the resonator might be
able to exert a one-sided, i.~e.~irreversible, action on the
energy of the surrounding radiation field called forth the
emphatic protest of Ludwig Boltzmann(5),
who with his more mature experience in these questions succeeded in
showing that according to the laws of the classical
dynamics every one of the processes I was considering
could take place in exactly the opposite sense. Thus
a spherical wave emitted from a resonator when reversed
shrinks in concentric spherical surfaces of continually decreasing
size on to the resonator, is absorbed by it, and so
permits the resonator to send out again into space the
energy formerly absorbed in the direction from which it
came. And although I was able to exclude such singular
processes as inwardly directed spherical waves by the
introduction of a special restriction, to wit the hypothesis
of `natural radiation', yet in the course of these investigations
it became more and more evident that in the chain
of argument an essential link was missing which should
lead to the comprehension of the nature of the entire
question.

The only way out of the difficulty was to attack the
problem from the opposite side, from the standpoint of
thermodynamics, a domain in which I felt more at home.
And as a matter of fact my previous studies on the second
law of thermodynamics served me here in good stead, in
that my first impulse was to bring not the temperature but
the entropy of the resonator into relation with its energy,
more accurately not the entropy itself but its second
derivative with respect to the energy, for it is this
differential coefficient that has a direct physical significance
for the irreversibility of the exchange of energy between
the resonator and the radiation. But as I was at that time
too much devoted to pure phenomenology to inquire more
closely into the relation between entropy and probability,
I felt compelled to limit myself to the available experimental
results. Now, at that time, in 1899, interest
was centred on the law of the distribution of energy,
which had not long before been proposed by
W.~Wien(6), the experimental verification of which had been undertaken
by F.~Paschen in Hanover and by O.~Lummer and
E.~Pringsheim of the Reichsanstalt, Charlottenburg. This
law expresses the intensity of radiation in terms of the
temperature by means of an exponential function. On
calculating the relation following from this law between
the entropy and energy of a resonator the remarkable
result is obtained that the reciprocal value of the above
differential coefficient, which I shall here denote by $R$, is
proportional to the energy(7). This extremely simple relation
can be regarded as an adequate expression of
Wien's law of the distribution of energy; for with the dependence
on the energy that of the wave-length is always
directly given by the well-established displacement law of
Wien(8).

Since this whole problem deals with a universal law of
nature, and since I was then, as to-day, pervaded with
a view that the more general and natural a law is the
simpler it is (although the question as to which formulation
is to be regarded as the simpler cannot always be definitely
and unambiguously decided), I believed for the time that
the basis of the law of the distribution of energy could
be expressed by the theorem that the value of $R$ is proportional
to the energy(9). But in view of the results of new measurements
this conception soon proved untenable.
For while Wien's law was completely satisfactory
for small values of energy and for short waves, on the one
hand it was shown by O.~Lummer and E.~Pringsheim
that considerable deviations were obtained with longer
waves(10), and on the other hand the measurements carried
out by H.~Rubens and F.~Kurlbaum with the infra-red
residual rays (\textit{Reststrahlen}) of fluorspar and rock
salt(11) disclosed a totally different, but, under certain circumstances,
a very simple relation characterized by the proportionality
of the value of $R$ not to the energy but to the
square of the energy. The longer the waves and the greater
the energy(12) the more accurately did this relation hold.

Thus two simple limits were established by direct
observation for the function $R$: for small energies proportionality
to the energy, for large energies proportionality to
the square of the energy. Nothing therefore seemed
simpler than to put in the general case $R$ equal to the sum
of a term proportional to the first power and another
proportional to the square of the energy, so that the first
term is relevant for small energies and the second for large
energies; and thus was found a new radiation formula(13)
which up to the present has withstood experimental
examination fairly satisfactorily. Nevertheless it cannot
be regarded as having been experimentally confirmed with
final accuracy, and a renewed test would be most
desirable(14).

But even if this radiation formula should prove to be
absolutely accurate it would after all be only an interpolation
formula found by happy guesswork, and would thus
leave one rather unsatisfied. I was, therefore, from the
day of its origination, occupied with the task of giving it
a real physical meaning, and this question led me, along
Boltzmann's line of thought, to the consideration of the
relation between entropy and probability; until after some
weeks of the most intense work of my life clearness began
to dawn upon me, and an unexpected view revealed itself
in the distance.

Let me here make a small digression. Entropy,
according to Boltzmann, is a measure of a physical probability,
and the meaning of the second law of thermodynamics
is that the more probable a state is, the more
frequently will it occur in nature. Now what one measures
are only the differences of entropy, and never entropy
itself, and consequently one cannot speak, in a definite
way, of the absolute entropy of a state. But nevertheless
the introduction of an appropriately defined absolute
magnitude of entropy is to be recommended, for the reason
that by its help certain general laws can be formulated
with great simplicity. As far as I can see the case is here
the same as with energy. Energy, too, cannot itself be
measured; only its differences can. In fact, the concept
used by our predecessors was not energy but work, and
even Ernst Mach, who devoted much attention to the law
of conservation of energy but at the same time strictly
avoided all speculations exceeding the limits of observation,
always abstained from speaking of energy itself. Similarly
in the early days of thermochemistry one was content to
deal with heats of reaction, that is to say again with
differences of energy, until Wilhelm Ostwald emphasized
that many complicated calculations could be materially
shortened if energies instead of calorimetric numbers were
used. The additive constant which thus remained undetermined
for energy was later finally fixed by the
relativistic law of the proportionality between energy and
inertia(15).

As in the case of energy, it is now possible to define
an absolute value of entropy, and thus of physical probability,
by fixing the additive constant so that together
with the energy (or better still, the temperature) the entropy
also should vanish. Such considerations led to a comparatively
simple method of calculating the physical probability
of a given distribution of energy in a system of resonators,
which yielded precisely the same expression for entropy as
that corresponding to the radiation law(16); and it gave me
particular satisfaction, in compensation for the many
disappointments I had encountered, to learn from Ludwig
Boltzmann of his interest and entire acquiescence in my
new line of reasoning.

To work out these probability considerations the knowledge
of two universal constants is required, each of which
has an independent meaning, so that the evaluation of
these constants from the radiation law could serve as an
a posteriori test whether the whole process is merely
a mathematical artifice or has a true physical meaning.
The first constant is of a somewhat formal nature; it is
connected with the definition of temperature. If temperature
were defined as the mean kinetic energy of a molecule
in a perfect gas, which is a minute energy indeed, this
constant would have the value $\frac{2}{3}$(17). But in the conventional
scale of temperature the constant assumes
(instead of $\frac{2}{3}$) an extremely small value, which naturally is
intimately connected with the energy of a single molecule,
so that its accurate determination would lead to the
calculation of the mass of a molecule and of associated
magnitudes. This constant is frequently termed Boltzmann's
constant, although to the best of my knowledge
Boltzmann himself never introduced it (an odd circumstance,
which no doubt can be explained by the fact that
he, as appears from certain of his statements(18), never
believed it would be possible to determine this constant
accurately). Nothing can better illustrate the rapid
progress of experimental physics within the last twenty
years than the fact that during this period not only one,
but a host of methods have been discovered by means of
which the mass of a single molecule can be measured with
almost the same accuracy as that of a planet.

While at the time when I carried out this calculation on
the basis of the radiation law an exact test of the value thus
obtained was quite impossible, and one could scarcely hope
to do more than test the admissibility of its order of
magnitude, it was not long before E.~Rutherford and
H.~Geiger(19) succeeded, by means of a direct count of the
$\alpha$-particles, in determining the value of the electrical elementary
charge as $4.65 \cdot 10^{-10}$, the agreement of which with
my value $4.69 \cdot 10^{-10}$ could be regarded as a decisive confirmation
of my theory. Since then further methods have
been developed by E.~Regener, R.~A.~Millikan, and others(20),
which have led to a but slightly higher value.

Much less simple than that of the first was the interpretation
of the second universal constant of the radiation law,
which, as the product of energy and time (amounting on a
first calculation to $6.55 \cdot 10^{-27}$ erg$\cdot$sec.) I called the elementary
quantum of action. While this constant was absolutely
indispensable to the attainment of a correct expression
for entropy---for only with its aid could be determined the
magnitude of the `elementary region' or `range' of probability,
necessary for the statistical treatment of the
problem(21)---it obstinately withstood all attempts at fitting
it, in any suitable form, into the frame of the classical
theory. So long as it could be regarded as infinitely small,
that is to say for large values of energy or long periods of
time, all went well; but in the general case a difficulty
arose at some point or other, which became the more pronounced
the weaker and the more rapid the oscillations.
The failure of all attempts to bridge this gap soon placed
one before the dilemma: either the quantum of action was
only a fictitious magnitude, and, therefore, the entire deduction
from the radiation law was illusory and a mere
juggling with formulae, or there is at the bottom of this
method of deriving the radiation law some true physical
concept. If the latter were the case, the quantum would
have to play a fundamental r\^{o}le in physics, heralding the
advent of a new state of things, destined, perhaps, to transform
completely our physical concepts which since the
introduction of the infinitesimal calculus by Leibniz and
Newton have been founded upon the assumption of the
continuity of all causal chains of events.

Experience has decided for the second alternative. But
that the decision should come so soon and so unhesitatingly
was due not to the examination of the law of distribution
of the energy of heat radiation, still less to my special
deduction of this law, but to the steady progress of the
work of those investigators who have applied the concept
of the quantum of action to their researches.

\label{Einstein}
The first advance in this field was made by A.~Einstein,
who on the one hand pointed out that the introduction of
the quanta of energy associated with the quantum of action
seemed capable of explaining readily a series of remarkable
properties of light action discovered experimentally, such
as Stokes's rule, the emission of electrons, and the ionization
of gases(22), and on the other hand, by the identification
of the expression for the energy of a system of resonators
with the energy of a solid body, derived a formula for the
specific heat of solid bodies which on the whole represented
it correctly as a function of temperature, more especially
exhibiting its decrease with falling temperature(23). A
number of questions were thus thrown out in different
directions, of which the accurate and many-sided investigations
yielded in the course of time much valuable material.
It is not my task to-day to give an even approximately
complete report of the successful work achieved in this
field; suffice it to give the most important and characteristic
phase of the progress of the new doctrine.

First, as to thermal and chemical processes. With regard
to specific heat of solid bodies, Einstein's view, which rests
on the assumption of a single free period of the atoms, was
extended by M.~Born and Th.~von~Karman to the case
which corresponds better to reality, viz. that of several free
periods(24); while P.~Debye, by a bold simplification of
the assumptions as to the nature of the free periods, succeeded
in developing a comparatively simple formula for
the specific heat of solid bodies(25) which excellently represents
its values, especially those for low temperatures
obtained by W.~Nernst and his pupils, and which, moreover,
is compatible with the elastic and optical properties of such
bodies. But the influence of the quanta asserts itself also
in the case of the specific heat of gases. At the very
outset it was pointed out by W.~Nernst(26) that to the
energy quantum of vibration must correspond an energy
quantum of rotation, and it was therefore to be expected
that the rotational energy of gas molecules would also
vanish at low temperatures. This conclusion was confirmed
by measurements, due to A.~Eucken, of the specific heat of
hydrogen(27); and if the calculations of A.~Einstein and
O.~Stern, P.~Ehrenfest, and others have not as yet yielded
completely satisfactory agreement, this no doubt is due to
our imperfect knowledge of the structure of the hydrogen
atom. That `quantized' rotations of gas molecules (i.~e.~satisfying
the quantum condition) do actually occur in
nature can no longer be doubted, thanks to the work on
absorption bands in the infra-red of N.~Bjerrum, E.~v.~Bahr,
H.~Rubens and G.~Hettner, and others, although a completely
exhaustive explanation of their remarkable rotation
spectra is still outstanding.

Since all affinity properties of a substance are ultimately
determined by its entropy, the quantic calculation of entropy
also gives access to all problems of chemical affinity.
The absolute value of the entropy of a gas is characterized
by Nernst's chemical constant, which was calculated by
O.~Sackur by a straightforward combinatorial process similar
to that applied to the case of the oscillators(28), while
H.~Tetrode, holding more closely to experimental data,
determined, by a consideration of the process of vaporization,
the difference of entropy between a substance and its
vapour(29).

While the cases thus far considered have dealt with
states of thermodynamical equilibrium, for which the measurements
could yield only statistical averages for large
numbers of particles and for comparatively long periods of
time, the observation of the collisions of electrons leads
directly to the dynamic details of the processes in question.
Therefore the determination, carried out by J.~Franck and
G.~Hertz, of the so-called resonance potential or the critical
velocity which an electron impinging upon a neutral atom
must have in order to cause it to emit a quantum of light,
provides a most direct method for the measurement of the
quantum of action(30). Similar methods leading to perfectly
consistent results can also be developed for the
excitation of the characteristic X-ray radiation discovered
by C.~G.~Barkla, as can be judged from the experiments
of D.~L.~Webster, E.~Wagner, and others.

The inverse of the process of producing light quanta by
the impact of electrons is the emission of electrons on
exposure to light-rays, or X-rays, and here, too, the energy
quanta following from the action quantum and the vibration
period play a characteristic r\^{o}le, as was early recognized
from the striking fact that the velocity of the emitted
electrons depends not upon the intensity(31) but only on
the colour of the impinging light(32). But quantitatively
also the relations to the light quantum, pointed out by
Einstein (p.~\pageref{Einstein}), %% adjust for generated pagination
have proved successful in every direction,
as was shown especially by R.~A.~Millikan, by measurements
of the velocities of emission of electrons(33), while
the importance of the light quantum in inducing photochemical
reactions was disclosed by E.~Warburg(34).

Although the results I have hitherto quoted from the most
diverse chapters of physics, taken in their totality, form an
overwhelming proof of the existence of the quantum of
action, the quantum hypothesis received its strongest support
from the theory of the structure of atoms (Quantum
Theory of Spectra) proposed and developed by Niels Bohr.
For it was the lot of this theory to find the long-sought key
to the gates of the wonderland of spectroscopy which since
the discovery of spectrum analysis up to our days had stubbornly
refused to yield. And the way once clear, a stream
of new knowledge poured in a sudden flood, not only over
this entire field but into the adjacent territories of physics
and chemistry. Its first brilliant success was the derivation
of Balmer's formula for the spectrum series of hydrogen and
helium, together with the reduction of the universal constant
of Rydberg to known magnitudes(35); and even the
small differences of the Rydberg constant for these two
gases appeared as a necessary consequence of the slight
wobbling of the massive atomic nucleus (accompanying the
motion of electrons around it). As a sequel came the
investigation of other series in the visual and especially
the X-ray spectrum aided by Ritz's resourceful combination
principle, which only now was recognized in its fundamental
significance.

But whoever may have still felt inclined, even in the
face of this almost overwhelming agreement---all the more
convincing, in view of the extreme accuracy of spectroscopic
measurements---to believe it to be a coincidence,
must have been compelled to give up his last doubt when
A.~Sommerfeld deduced, by a logical extension of the laws
of the distribution of quanta in systems with several degrees
of freedom, and by a consideration of the variability of
inert mass required by the principle of relativity, that
magic formula before which the spectra of both hydrogen
and helium revealed the mystery of their `fine structure'(36),
as far as this could be disclosed by the most delicate
measurements possible up to the present, those of
F.~Paschen(37)---a success equal to the famous discovery
of the planet Neptune, the presence and orbit of which
were calculated by Leverrier [and Adams] before man
ever set eyes upon it. Progressing along the same road,
P.~Epstein achieved a complete explanation of the Stark effect
of the electrical splitting of spectral lines(38), P.~Debye obtained
a simple interpretation of the K-series(39) of the X-ray
spectrum investigated by Manne Siegbahn, and then followed
a long series of further researches which illuminated with
greater or less success the dark secret of atomic structure.

After all these results, for the complete exposition of
which many famous names would here have to be mentioned,
there must remain for an observer, who does not
choose to pass over the facts, no other conclusion than that
the quantum of action, which in every one of the many
and most diverse processes has always the same value,
namely $6.52 \cdot 10^{-27}$ erg$\cdot$sec.(40), deserves to be definitely
incorporated into the system of the universal physical constants.
It must certainly appear a strange coincidence that
at just the same time as the idea of general relativity arose
and scored its first great successes, nature revealed, precisely
in a place where it was the least to be expected, an
absolute and strictly unalterable unit, by means of which
the amount of action contained in a space-time element can
be expressed by a perfectly definite number, and thus is
deprived of its former relative character.

Of course the mere introduction of the quantum of action
does not yet mean that a true Quantum Theory has been
established. Nay, the path which research has yet to cover
to reach that goal is perhaps not less long than that from
the discovery of the velocity of light by Olaf R\"{o}mer to the
foundation of Maxwell's theory of light. The difficulties
which the introduction of the quantum of action into the
well-established classical theory has encountered from the
outset have already been indicated. They have gradually
increased rather than diminished; and although research
in its forward march has in the meantime passed over
some of them, the remaining gaps in the theory are the
more distressing to the conscientious theoretical physicist.
In fact, what in Bohr's theory served as the basis of the
laws of action consists of certain hypotheses which a generation
ago would doubtless have been flatly rejected by
every physicist. That with the atom certain quantized
orbits [i.~e. picked out on the quantum principle] should play
a special r\^{o}le could well be granted; somewhat less easy
to accept is the further assumption that the electrons
moving on these curvilinear orbits, and therefore accelerated,
radiate no energy. But that the sharply defined
frequency of an emitted light quantum should be different
from the frequency of the emitting electron would be regarded
by a theoretician who had grown up in the classical
school as monstrous and almost inconceivable.

But numbers decide, and in consequence the tables have
been turned. While originally it was a question of fitting
in with as little strain as possible a new and strange element
into an existing system which was generally regarded
as settled, the intruder, after having won an assured position,
now has assumed the offensive; and it now appears
certain that it is about to blow up the old system at some
point. The only question now is, at what point and to
what extent this will happen. If I may express at the
present time a conjecture as to the probable outcome of
this desperate struggle, everything appears to indicate that
out of the classical theory the great principles of thermodynamics
will not only maintain intact their central position
in the quantum theory, but will perhaps even extend their
influence. The significant part played in the origin of the
classical thermodynamics by mental experiments is now
taken over in the quantum theory by P.~Ehrenfest's hypothesis
of the adiabatic invariance(41); and just as the
principle introduced by R.~Clausius, that any two states of
a material system are mutually interconvertible on suitable
treatment by reversible processes, formed the basis for the
measurement of entropy, just so do the new ideas of Bohr
show a way into the midst of the wonderland he has
discovered.

There is one particular question the answer to which
will, in my opinion, lead to an extensive elucidation of the
entire problem. What happens to the energy of a light-quantum
after its emission? Does it pass outwards in all
directions, according to Huygens's wave theory, continually
increasing in volume and tending towards infinite dilution?
Or does it, as in Newton's emanation theory, fly like a projectile
in one direction only? In the former case the
quantum would never again be in a position to concentrate
its energy at a spot strongly enough to detach an electron
from its atom; while in the latter case it would be necessary
to sacrifice the chief triumph of Maxwell's theory---the
continuity between the static and the dynamic fields---and
with it the classical theory of the interference phenomena
which accounted for all their details, both alternatives
leading to consequences very disagreeable to the modern
theoretical physicist.

Whatever the answer to this question, there can be no
doubt that science will some day master the dilemma, and
what may now appear to us unsatisfactory will appear from
a higher standpoint as endowed with a particular harmony
and simplicity. But until this goal is reached the problem
of the quantum of action will not cease to stimulate
research, and the greater the difficulties encountered in
its solution the greater will be its significance for the
broadening and deepening of all our physical knowledge.

\clearpage

\begin{center}\Large{NOTES}\end{center}

The references to the literature are not claimed to be in any way
complete, and are intended to serve only for a preliminary orientation.

(1) G.~Kirchhoff, \"{U}ber das Verh\"{a}ltnis zwischen dem Emissionsverm\"{o}gen und dem Absorptionsverm\"{o}gen
der K\"{o}rper f\"{u}r\hspace{4pt}W\"{a}rme und Licht.
\textit{Gesammelte Abhandlungen.} Leipzig, J.~A.~Barth, 1882, p.~597 (\S 17).

(2) H.~Hertz, \textit{Ann.\ d.\ Phys.} \textbf{36}, p.~1, 1889.

(3) \textit{Sitz.-Ber.\ d.\ Preuss.\ Akad.\ d.\ Wiss.} Febr.\ 20, 1896. \textit{Ann.\ d.\ Phys.} \textbf{60}, p.\ 577, 1897.

(4) \textit{Sitz.-Ber.\ d.\ Preuss.\ Akad.\ d.\ Wiss.} May 18, 1899, p.\ 455.

(5) L.~Boltzmann, \textit{Sitz.-Ber.\ d.\ Preuss.\ Akad.\ d.\ Wiss.} March 3, 1898, p.\ 182.

(6) W.~Wien, \textit{Ann.\ d.\ Phys.} \textbf{58}, p.\ 662, 1896.

(7) According to Wien's law of the distribution of energy the
dependence of the energy $U$ of the resonator upon the temperature
is given by a relation of the form:
$$U=a \cdot e^{-b/t}.$$
Since
$$\frac{1}{T}=\frac{dS}{dU},$$
where $S$ is the entropy of the resonator, we have for $R$ as used in the text:
$$R=1:\frac{d^2S}{dU^2}=-bU.$$

(8) According to Wien's displacement law, the energy $U$ of the resonator with the
natural vibration period $\nu$ is expressed by: $$U=\nu \cdot f\left(\frac{T}{\nu}\right).$$

(9) \textit{Ann.\ d.\ Phys.} \textbf{1}, p.\ 719, 1900.

(10) O.~Lummer und E.~Pringsheim, \textit{Verhandl.\ der Deutschen Physikal.\ Ges.}, \textbf{2}, p.\ 163, 1900.

(11) H.~Rubens and F.~Kurlbaum, \textit{Sitz.-Ber.\ der Preuss.\ Akad.\ d.\ Wiss.} Oct.\ 25, 1900, p.\ 929.

(12) It follows from the experiments of H.~Rubens and F.~Kurlbaum that, for high temperatures, $U=cT$.
Then, in accordance with the method quoted in (7): $$R=1:\frac{d^2S}{dU^2}=-\frac{U^2}{c}.$$

(13) Put $$R=1:\frac{d^2S}{dU^2}=-bU-\frac{U^2}{c},$$
then by integration, $$\frac{1}{T}=\frac{dS}{dU}=\frac{1}{b}\log \left\{ 1+\frac{bc}{U}\right\} $$
whence the radiation formula, $$U=bc:(e^{-b/T}-1).$$
Cf.\ \textit{Verhandlungen der Deutschen Phys.\ Ges.} Oct.\ 19, 1900, p.\ 202.

(14) Cf.\ W.~Nernst und Th.~Wulf, \textit{Verh.\ d.\ Deutsch.\ Phys.\ Ges.} \textbf{21},
p.\ 294, 1919.

(15) For the absolute value of the energy is equal to the product
of the inert mass and the square of light velocity.

(16) \textit{Verhandlungen der Deutschen Phys.\ Ges.} Dec.\ 14, 1900, p.\ 237.

(17) Generally, if $k$ be the first radiation constant, the mean kinetic
energy of a gas molecule is:$$U=\frac{3}{2}kT$$

If we put, therefore, $T=U$, then $k=\frac{2}{3}$. In the conventional [absolute
Kelvinian] temperature scale, however, $T$ is defined by putting the
temperature difference between boiling and freezing water equal to 100.

(18) Cf.\ for example L.~Boltzmann, Zur Erinnerung an Josef Loschmidt, \textit{Popul\"{a}re Schriften}, p.\ 245, 1905.

(19) E.~Rutherford and H.~Geiger, \textit{Proc.\ Roy.\ Soc.} A.\ Vol.\ \textbf{81}, p.\ 162, 1908.

(20) Cf.\ R.~A.~Millikan, \textit{Phys.\ Zeitschr.} \textbf{14}, p.\ 796, 1913.

(21) The evaluation of the probability of a physical state is based
upon counting that finite number of equally probable special cases
by which the corresponding state is realized; and in order sharply
to distinguish these cases from one another, a definite concept of each
special case has necessarily to be introduced.

(22) A.~Einstein, \textit{Ann.\ d.\ Phys.} \textbf{17}, p.\ 132, 1905.

(23) A.~Einstein, \textit{Ann.\ d.\ Phys.} \textbf{22}, p.\ 180, 1907.

(24) M.~Born und Th.~v.~Karman, \textit{Phys.\ Zeitschr.} \textbf{14}, p.\ 15, 1913.

(25) P.~Debye, \textit{Ann.\ d.\ Phys.} \textbf{39}, p.\ 789, 1912.

(26) W.~Nernst, \textit{Phys.\ Zeitschr.} \textbf{13}, p.\ 1064, 1912.

(27) A.~Eucken, \textit{Sitz.-Ber.\ d.\ Preuss.\ Akad.\ d.\ Wiss.} p.\ 141, 1912.

(28) O.~Sackur, \textit{Ann.\ d.\ Phys.} \textbf{36}, p.\ 958, 1911.

(29) H.~Tetrode, \textit{Proc.\ Acad.\ Sci.\ Amsterdam}, Febr.\ 27 and March 27, 1915.

(30) J.~Franck und G.~Hertz, \textit{Verh.\ d.\ Deutsch.\ Phys.\ Ges.} \textbf{16}, p.\ 512, 1914.

(31) Ph.~Lenard, \textit{Ann.\ d.\ Phys.} \textbf{8}, p.\ 149, 1902.

(32) E.~Ladenburg, \textit{Verh.\ d.\ Deutschen Phys.\ Ges.} \textbf{9}, p.\ 504, 1907.

(33) R.~A.~Millikan, \textit{Phys.\ Zeitschr.} \textbf{17}, p.\ 217, 1916.

(34) E.~Warburg, \"{U}ber den Energieumsatz bei photochemischen Vorg\"{a}ngen in Gasen.
\textit{Sitz.-Ber.\ d.\ Preuss.\ Akad.\ d.\ Wiss.} from 1911 onwards.

(35) N.~Bohr, \textit{Phil.\ Mag.} \textbf{30}, p.\ 394, 1915.

(36) A.~Sommerfeld, \textit{Ann.\ d.\ Phys.} \textbf{51}, pp.\ 1, 125, 1916.

(37) F.~Paschen, \textit{Ann.\ d.\ Phys.} \textbf{50}, p.\ 901, 1916.

(38) P.~Epstein, \textit{Ann.\ d.\ Phys.} \textbf{50}, p.\ 489, 1916.

(39) P.~Debye, \textit{Phys.\ Zeitschr.} \textbf{18}, p.\ 276, 1917.

(40) E.~Wagner, \textit{Ann.\ d.\ Phys.} \textbf{57}, p.\ 467, 1918.

(41) P.~Ehrenfest, \textit{Ann.\ d.\ Phys.} \textbf{51}, p.\ 327, 1916.

\clearpage
\pagenumbering{Alph}

\small
\begin{verbatim}
End of the Project Gutenberg EBook of The Origin and Development of the
Quantum Theory, by Max Planck

*** END OF THIS PROJECT GUTENBERG EBOOK QUANTUM THEORY ***

***** This file should be named 33663-pdf.pdf or 33663-pdf.zip *****
This and all associated files of various formats will be found in:
        http://www.gutenberg.org/3/3/6/6/33663/

Produced by Roger Frank and the Online Distributed
Proofreading Team at http://www.fadedpage.net


Updated editions will replace the previous one--the old editions
will be renamed.

Creating the works from public domain print editions means that no
one owns a United States copyright in these works, so the Foundation
(and you!) can copy and distribute it in the United States without
permission and without paying copyright royalties.  Special rules,
set forth in the General Terms of Use part of this license, apply to
copying and distributing Project Gutenberg-tm electronic works to
protect the PROJECT GUTENBERG-tm concept and trademark.  Project
Gutenberg is a registered trademark, and may not be used if you
charge for the eBooks, unless you receive specific permission.  If you
do not charge anything for copies of this eBook, complying with the
rules is very easy.  You may use this eBook for nearly any purpose
such as creation of derivative works, reports, performances and
research.  They may be modified and printed and given away--you may do
practically ANYTHING with public domain eBooks.  Redistribution is
subject to the trademark license, especially commercial
redistribution.



*** START: FULL LICENSE ***

THE FULL PROJECT GUTENBERG LICENSE
PLEASE READ THIS BEFORE YOU DISTRIBUTE OR USE THIS WORK

To protect the Project Gutenberg-tm mission of promoting the free
distribution of electronic works, by using or distributing this work
(or any other work associated in any way with the phrase "Project
Gutenberg"), you agree to comply with all the terms of the Full Project
Gutenberg-tm License (available with this file or online at
http://gutenberg.org/license).


Section 1.  General Terms of Use and Redistributing Project Gutenberg-tm
electronic works

1.A.  By reading or using any part of this Project Gutenberg-tm
electronic work, you indicate that you have read, understand, agree to
and accept all the terms of this license and intellectual property
(trademark/copyright) agreement.  If you do not agree to abide by all
the terms of this agreement, you must cease using and return or destroy
all copies of Project Gutenberg-tm electronic works in your possession.
If you paid a fee for obtaining a copy of or access to a Project
Gutenberg-tm electronic work and you do not agree to be bound by the
terms of this agreement, you may obtain a refund from the person or
entity to whom you paid the fee as set forth in paragraph 1.E.8.

1.B.  "Project Gutenberg" is a registered trademark.  It may only be
used on or associated in any way with an electronic work by people who
agree to be bound by the terms of this agreement.  There are a few
things that you can do with most Project Gutenberg-tm electronic works
even without complying with the full terms of this agreement.  See
paragraph 1.C below.  There are a lot of things you can do with Project
Gutenberg-tm electronic works if you follow the terms of this agreement
and help preserve free future access to Project Gutenberg-tm electronic
works.  See paragraph 1.E below.

1.C.  The Project Gutenberg Literary Archive Foundation ("the Foundation"
or PGLAF), owns a compilation copyright in the collection of Project
Gutenberg-tm electronic works.  Nearly all the individual works in the
collection are in the public domain in the United States.  If an
individual work is in the public domain in the United States and you are
located in the United States, we do not claim a right to prevent you from
copying, distributing, performing, displaying or creating derivative
works based on the work as long as all references to Project Gutenberg
are removed.  Of course, we hope that you will support the Project
Gutenberg-tm mission of promoting free access to electronic works by
freely sharing Project Gutenberg-tm works in compliance with the terms of
this agreement for keeping the Project Gutenberg-tm name associated with
the work.  You can easily comply with the terms of this agreement by
keeping this work in the same format with its attached full Project
Gutenberg-tm License when you share it without charge with others.

1.D.  The copyright laws of the place where you are located also govern
what you can do with this work.  Copyright laws in most countries are in
a constant state of change.  If you are outside the United States, check
the laws of your country in addition to the terms of this agreement
before downloading, copying, displaying, performing, distributing or
creating derivative works based on this work or any other Project
Gutenberg-tm work.  The Foundation makes no representations concerning
the copyright status of any work in any country outside the United
States.

1.E.  Unless you have removed all references to Project Gutenberg:

1.E.1.  The following sentence, with active links to, or other immediate
access to, the full Project Gutenberg-tm License must appear prominently
whenever any copy of a Project Gutenberg-tm work (any work on which the
phrase "Project Gutenberg" appears, or with which the phrase "Project
Gutenberg" is associated) is accessed, displayed, performed, viewed,
copied or distributed:

This eBook is for the use of anyone anywhere at no cost and with
almost no restrictions whatsoever.  You may copy it, give it away or
re-use it under the terms of the Project Gutenberg License included
with this eBook or online at www.gutenberg.org

1.E.2.  If an individual Project Gutenberg-tm electronic work is derived
from the public domain (does not contain a notice indicating that it is
posted with permission of the copyright holder), the work can be copied
and distributed to anyone in the United States without paying any fees
or charges.  If you are redistributing or providing access to a work
with the phrase "Project Gutenberg" associated with or appearing on the
work, you must comply either with the requirements of paragraphs 1.E.1
through 1.E.7 or obtain permission for the use of the work and the
Project Gutenberg-tm trademark as set forth in paragraphs 1.E.8 or
1.E.9.

1.E.3.  If an individual Project Gutenberg-tm electronic work is posted
with the permission of the copyright holder, your use and distribution
must comply with both paragraphs 1.E.1 through 1.E.7 and any additional
terms imposed by the copyright holder.  Additional terms will be linked
to the Project Gutenberg-tm License for all works posted with the
permission of the copyright holder found at the beginning of this work.

1.E.4.  Do not unlink or detach or remove the full Project Gutenberg-tm
License terms from this work, or any files containing a part of this
work or any other work associated with Project Gutenberg-tm.

1.E.5.  Do not copy, display, perform, distribute or redistribute this
electronic work, or any part of this electronic work, without
prominently displaying the sentence set forth in paragraph 1.E.1 with
active links or immediate access to the full terms of the Project
Gutenberg-tm License.

1.E.6.  You may convert to and distribute this work in any binary,
compressed, marked up, nonproprietary or proprietary form, including any
word processing or hypertext form.  However, if you provide access to or
distribute copies of a Project Gutenberg-tm work in a format other than
"Plain Vanilla ASCII" or other format used in the official version
posted on the official Project Gutenberg-tm web site (www.gutenberg.org),
you must, at no additional cost, fee or expense to the user, provide a
copy, a means of exporting a copy, or a means of obtaining a copy upon
request, of the work in its original "Plain Vanilla ASCII" or other
form.  Any alternate format must include the full Project Gutenberg-tm
License as specified in paragraph 1.E.1.

1.E.7.  Do not charge a fee for access to, viewing, displaying,
performing, copying or distributing any Project Gutenberg-tm works
unless you comply with paragraph 1.E.8 or 1.E.9.

1.E.8.  You may charge a reasonable fee for copies of or providing
access to or distributing Project Gutenberg-tm electronic works provided
that

- You pay a royalty fee of 20% of the gross profits you derive from
     the use of Project Gutenberg-tm works calculated using the method
     you already use to calculate your applicable taxes.  The fee is
     owed to the owner of the Project Gutenberg-tm trademark, but he
     has agreed to donate royalties under this paragraph to the
     Project Gutenberg Literary Archive Foundation.  Royalty payments
     must be paid within 60 days following each date on which you
     prepare (or are legally required to prepare) your periodic tax
     returns.  Royalty payments should be clearly marked as such and
     sent to the Project Gutenberg Literary Archive Foundation at the
     address specified in Section 4, "Information about donations to
     the Project Gutenberg Literary Archive Foundation."

- You provide a full refund of any money paid by a user who notifies
     you in writing (or by e-mail) within 30 days of receipt that s/he
     does not agree to the terms of the full Project Gutenberg-tm
     License.  You must require such a user to return or
     destroy all copies of the works possessed in a physical medium
     and discontinue all use of and all access to other copies of
     Project Gutenberg-tm works.

- You provide, in accordance with paragraph 1.F.3, a full refund of any
     money paid for a work or a replacement copy, if a defect in the
     electronic work is discovered and reported to you within 90 days
     of receipt of the work.

- You comply with all other terms of this agreement for free
     distribution of Project Gutenberg-tm works.

1.E.9.  If you wish to charge a fee or distribute a Project Gutenberg-tm
electronic work or group of works on different terms than are set
forth in this agreement, you must obtain permission in writing from
both the Project Gutenberg Literary Archive Foundation and Michael
Hart, the owner of the Project Gutenberg-tm trademark.  Contact the
Foundation as set forth in Section 3 below.

1.F.

1.F.1.  Project Gutenberg volunteers and employees expend considerable
effort to identify, do copyright research on, transcribe and proofread
public domain works in creating the Project Gutenberg-tm
collection.  Despite these efforts, Project Gutenberg-tm electronic
works, and the medium on which they may be stored, may contain
"Defects," such as, but not limited to, incomplete, inaccurate or
corrupt data, transcription errors, a copyright or other intellectual
property infringement, a defective or damaged disk or other medium, a
computer virus, or computer codes that damage or cannot be read by
your equipment.

1.F.2.  LIMITED WARRANTY, DISCLAIMER OF DAMAGES - Except for the "Right
of Replacement or Refund" described in paragraph 1.F.3, the Project
Gutenberg Literary Archive Foundation, the owner of the Project
Gutenberg-tm trademark, and any other party distributing a Project
Gutenberg-tm electronic work under this agreement, disclaim all
liability to you for damages, costs and expenses, including legal
fees.  YOU AGREE THAT YOU HAVE NO REMEDIES FOR NEGLIGENCE, STRICT
LIABILITY, BREACH OF WARRANTY OR BREACH OF CONTRACT EXCEPT THOSE
PROVIDED IN PARAGRAPH 1.F.3.  YOU AGREE THAT THE FOUNDATION, THE
TRADEMARK OWNER, AND ANY DISTRIBUTOR UNDER THIS AGREEMENT WILL NOT BE
LIABLE TO YOU FOR ACTUAL, DIRECT, INDIRECT, CONSEQUENTIAL, PUNITIVE OR
INCIDENTAL DAMAGES EVEN IF YOU GIVE NOTICE OF THE POSSIBILITY OF SUCH
DAMAGE.

1.F.3.  LIMITED RIGHT OF REPLACEMENT OR REFUND - If you discover a
defect in this electronic work within 90 days of receiving it, you can
receive a refund of the money (if any) you paid for it by sending a
written explanation to the person you received the work from.  If you
received the work on a physical medium, you must return the medium with
your written explanation.  The person or entity that provided you with
the defective work may elect to provide a replacement copy in lieu of a
refund.  If you received the work electronically, the person or entity
providing it to you may choose to give you a second opportunity to
receive the work electronically in lieu of a refund.  If the second copy
is also defective, you may demand a refund in writing without further
opportunities to fix the problem.

1.F.4.  Except for the limited right of replacement or refund set forth
in paragraph 1.F.3, this work is provided to you 'AS-IS' WITH NO OTHER
WARRANTIES OF ANY KIND, EXPRESS OR IMPLIED, INCLUDING BUT NOT LIMITED TO
WARRANTIES OF MERCHANTIBILITY OR FITNESS FOR ANY PURPOSE.

1.F.5.  Some states do not allow disclaimers of certain implied
warranties or the exclusion or limitation of certain types of damages.
If any disclaimer or limitation set forth in this agreement violates the
law of the state applicable to this agreement, the agreement shall be
interpreted to make the maximum disclaimer or limitation permitted by
the applicable state law.  The invalidity or unenforceability of any
provision of this agreement shall not void the remaining provisions.

1.F.6.  INDEMNITY - You agree to indemnify and hold the Foundation, the
trademark owner, any agent or employee of the Foundation, anyone
providing copies of Project Gutenberg-tm electronic works in accordance
with this agreement, and any volunteers associated with the production,
promotion and distribution of Project Gutenberg-tm electronic works,
harmless from all liability, costs and expenses, including legal fees,
that arise directly or indirectly from any of the following which you do
or cause to occur: (a) distribution of this or any Project Gutenberg-tm
work, (b) alteration, modification, or additions or deletions to any
Project Gutenberg-tm work, and (c) any Defect you cause.


Section  2.  Information about the Mission of Project Gutenberg-tm

Project Gutenberg-tm is synonymous with the free distribution of
electronic works in formats readable by the widest variety of computers
including obsolete, old, middle-aged and new computers.  It exists
because of the efforts of hundreds of volunteers and donations from
people in all walks of life.

Volunteers and financial support to provide volunteers with the
assistance they need, are critical to reaching Project Gutenberg-tm's
goals and ensuring that the Project Gutenberg-tm collection will
remain freely available for generations to come.  In 2001, the Project
Gutenberg Literary Archive Foundation was created to provide a secure
and permanent future for Project Gutenberg-tm and future generations.
To learn more about the Project Gutenberg Literary Archive Foundation
and how your efforts and donations can help, see Sections 3 and 4
and the Foundation web page at http://www.pglaf.org.


Section 3.  Information about the Project Gutenberg Literary Archive
Foundation

The Project Gutenberg Literary Archive Foundation is a non profit
501(c)(3) educational corporation organized under the laws of the
state of Mississippi and granted tax exempt status by the Internal
Revenue Service.  The Foundation's EIN or federal tax identification
number is 64-6221541.  Its 501(c)(3) letter is posted at
http://pglaf.org/fundraising.  Contributions to the Project Gutenberg
Literary Archive Foundation are tax deductible to the full extent
permitted by U.S. federal laws and your state's laws.

The Foundation's principal office is located at 4557 Melan Dr. S.
Fairbanks, AK, 99712., but its volunteers and employees are scattered
throughout numerous locations.  Its business office is located at
809 North 1500 West, Salt Lake City, UT 84116, (801) 596-1887, email
business@pglaf.org.  Email contact links and up to date contact
information can be found at the Foundation's web site and official
page at http://pglaf.org

For additional contact information:
     Dr. Gregory B. Newby
     Chief Executive and Director
     gbnewby@pglaf.org


Section 4.  Information about Donations to the Project Gutenberg
Literary Archive Foundation

Project Gutenberg-tm depends upon and cannot survive without wide
spread public support and donations to carry out its mission of
increasing the number of public domain and licensed works that can be
freely distributed in machine readable form accessible by the widest
array of equipment including outdated equipment.  Many small donations
($1 to $5,000) are particularly important to maintaining tax exempt
status with the IRS.

The Foundation is committed to complying with the laws regulating
charities and charitable donations in all 50 states of the United
States.  Compliance requirements are not uniform and it takes a
considerable effort, much paperwork and many fees to meet and keep up
with these requirements.  We do not solicit donations in locations
where we have not received written confirmation of compliance.  To
SEND DONATIONS or determine the status of compliance for any
particular state visit http://pglaf.org

While we cannot and do not solicit contributions from states where we
have not met the solicitation requirements, we know of no prohibition
against accepting unsolicited donations from donors in such states who
approach us with offers to donate.

International donations are gratefully accepted, but we cannot make
any statements concerning tax treatment of donations received from
outside the United States.  U.S. laws alone swamp our small staff.

Please check the Project Gutenberg Web pages for current donation
methods and addresses.  Donations are accepted in a number of other
ways including checks, online payments and credit card donations.
To donate, please visit: http://pglaf.org/donate


Section 5.  General Information About Project Gutenberg-tm electronic
works.

Professor Michael S. Hart is the originator of the Project Gutenberg-tm
concept of a library of electronic works that could be freely shared
with anyone.  For thirty years, he produced and distributed Project
Gutenberg-tm eBooks with only a loose network of volunteer support.


Project Gutenberg-tm eBooks are often created from several printed
editions, all of which are confirmed as Public Domain in the U.S.
unless a copyright notice is included.  Thus, we do not necessarily
keep eBooks in compliance with any particular paper edition.


Most people start at our Web site which has the main PG search facility:

     http://www.gutenberg.org

This Web site includes information about Project Gutenberg-tm,
including how to make donations to the Project Gutenberg Literary
Archive Foundation, how to help produce our new eBooks, and how to
subscribe to our email newsletter to hear about new eBooks.
\end{verbatim}

% %%%%%%%%%%%%%%%%%%%%%%%%%%%%%%%%%%%%%%%%%%%%%%%%%%%%%%%%%%%%%%%%%%%%%%% %
%                                                                         %
% End of the Project Gutenberg EBook of The Origin and Development of the %
% Quantum Theory, by Max Planck                                           %
%                                                                         %
% *** END OF THIS PROJECT GUTENBERG EBOOK QUANTUM THEORY ***              %
%                                                                         %
% ***** This file should be named 33663-t.tex or 33663-t.zip *****        %
% This and all associated files of various formats will be found in:      %
%         http://www.gutenberg.org/3/3/6/6/33663/                         %
%                                                                         %
% %%%%%%%%%%%%%%%%%%%%%%%%%%%%%%%%%%%%%%%%%%%%%%%%%%%%%%%%%%%%%%%%%%%%%%% %

\end{document}
###
###
This is pdfTeXk, Version 3.141592-1.40.3 (Web2C 7.5.6) (format=pdflatex 2010.5.6)  7 SEP 2010 04:30
entering extended mode
 %&-line parsing enabled.
**33663-t.tex
(./33663-t.tex
LaTeX2e <2005/12/01>
Babel <v3.8h> and hyphenation patterns for english, usenglishmax, dumylang, noh
yphenation, arabic, farsi, croatian, ukrainian, russian, bulgarian, czech, slov
ak, danish, dutch, finnish, basque, french, german, ngerman, ibycus, greek, mon
ogreek, ancientgreek, hungarian, italian, latin, mongolian, norsk, icelandic, i
nterlingua, turkish, coptic, romanian, welsh, serbian, slovenian, estonian, esp
eranto, uppersorbian, indonesian, polish, portuguese, spanish, catalan, galicia
n, swedish, ukenglish, pinyin, loaded.
(/usr/share/texmf-texlive/tex/latex/base/book.cls
Document Class: book 2005/09/16 v1.4f Standard LaTeX document class
(/usr/share/texmf-texlive/tex/latex/base/bk12.clo
File: bk12.clo 2005/09/16 v1.4f Standard LaTeX file (size option)
)
\c@part=\count79
\c@chapter=\count80
\c@section=\count81
\c@subsection=\count82
\c@subsubsection=\count83
\c@paragraph=\count84
\c@subparagraph=\count85
\c@figure=\count86
\c@table=\count87
\abovecaptionskip=\skip41
\belowcaptionskip=\skip42
\bibindent=\dimen102
) (/usr/share/texmf-texlive/tex/latex/geometry/geometry.sty
Package: geometry 2002/07/08 v3.2 Page Geometry
(/usr/share/texmf-texlive/tex/latex/graphics/keyval.sty
Package: keyval 1999/03/16 v1.13 key=value parser (DPC)
\KV@toks@=\toks14
)
\Gm@cnth=\count88
\Gm@cntv=\count89
\c@Gm@tempcnt=\count90
\Gm@bindingoffset=\dimen103
\Gm@wd@mp=\dimen104
\Gm@odd@mp=\dimen105
\Gm@even@mp=\dimen106
\Gm@dimlist=\toks15
(/usr/share/texmf-texlive/tex/xelatex/xetexconfig/geometry.cfg)) (/usr/share/te
xmf-texlive/tex/latex/base/fontenc.sty
Package: fontenc 2005/09/27 v1.99g Standard LaTeX package
(/usr/share/texmf-texlive/tex/latex/base/t1enc.def
File: t1enc.def 2005/09/27 v1.99g Standard LaTeX file
LaTeX Font Info:    Redeclaring font encoding T1 on input line 43.
)) (/usr/share/texmf/tex/latex/lm/lmodern.sty
Package: lmodern 2007/01/14 v1.3 Latin Modern Fonts
LaTeX Font Info:    Overwriting symbol font `operators' in version `normal'
(Font)                  OT1/cmr/m/n --> OT1/lmr/m/n on input line 13.
LaTeX Font Info:    Overwriting symbol font `letters' in version `normal'
(Font)                  OML/cmm/m/it --> OML/lmm/m/it on input line 14.
LaTeX Font Info:    Overwriting symbol font `symbols' in version `normal'
(Font)                  OMS/cmsy/m/n --> OMS/lmsy/m/n on input line 15.
LaTeX Font Info:    Overwriting symbol font `largesymbols' in version `normal'
(Font)                  OMX/cmex/m/n --> OMX/lmex/m/n on input line 16.
LaTeX Font Info:    Overwriting symbol font `operators' in version `bold'
(Font)                  OT1/cmr/bx/n --> OT1/lmr/bx/n on input line 17.
LaTeX Font Info:    Overwriting symbol font `letters' in version `bold'
(Font)                  OML/cmm/b/it --> OML/lmm/b/it on input line 18.
LaTeX Font Info:    Overwriting symbol font `symbols' in version `bold'
(Font)                  OMS/cmsy/b/n --> OMS/lmsy/b/n on input line 19.
LaTeX Font Info:    Overwriting symbol font `largesymbols' in version `bold'
(Font)                  OMX/cmex/m/n --> OMX/lmex/m/n on input line 20.
LaTeX Font Info:    Overwriting math alphabet `\mathbf' in version `normal'
(Font)                  OT1/cmr/bx/n --> OT1/lmr/bx/n on input line 22.
LaTeX Font Info:    Overwriting math alphabet `\mathsf' in version `normal'
(Font)                  OT1/cmss/m/n --> OT1/lmss/m/n on input line 23.
LaTeX Font Info:    Overwriting math alphabet `\mathit' in version `normal'
(Font)                  OT1/cmr/m/it --> OT1/lmr/m/it on input line 24.
LaTeX Font Info:    Overwriting math alphabet `\mathtt' in version `normal'
(Font)                  OT1/cmtt/m/n --> OT1/lmtt/m/n on input line 25.
LaTeX Font Info:    Overwriting math alphabet `\mathbf' in version `bold'
(Font)                  OT1/cmr/bx/n --> OT1/lmr/bx/n on input line 26.
LaTeX Font Info:    Overwriting math alphabet `\mathsf' in version `bold'
(Font)                  OT1/cmss/bx/n --> OT1/lmss/bx/n on input line 27.
LaTeX Font Info:    Overwriting math alphabet `\mathit' in version `bold'
(Font)                  OT1/cmr/bx/it --> OT1/lmr/bx/it on input line 28.
LaTeX Font Info:    Overwriting math alphabet `\mathtt' in version `bold'
(Font)                  OT1/cmtt/m/n --> OT1/lmtt/m/n on input line 29.
) (/usr/share/texmf-texlive/tex/latex/fancyhdr/fancyhdr.sty
\fancy@headwidth=\skip43
\f@ncyO@elh=\skip44
\f@ncyO@erh=\skip45
\f@ncyO@olh=\skip46
\f@ncyO@orh=\skip47
\f@ncyO@elf=\skip48
\f@ncyO@erf=\skip49
\f@ncyO@olf=\skip50
\f@ncyO@orf=\skip51
) (/usr/share/texmf-texlive/tex/latex/tools/verbatim.sty
Package: verbatim 2003/08/22 v1.5q LaTeX2e package for verbatim enhancements
\every@verbatim=\toks16
\verbatim@line=\toks17
\verbatim@in@stream=\read1
) (./33663-t.aux)
\openout1 = `33663-t.aux'.

LaTeX Font Info:    Checking defaults for OML/cmm/m/it on input line 83.
LaTeX Font Info:    ... okay on input line 83.
LaTeX Font Info:    Checking defaults for T1/cmr/m/n on input line 83.
LaTeX Font Info:    ... okay on input line 83.
LaTeX Font Info:    Checking defaults for OT1/cmr/m/n on input line 83.
LaTeX Font Info:    ... okay on input line 83.
LaTeX Font Info:    Checking defaults for OMS/cmsy/m/n on input line 83.
LaTeX Font Info:    ... okay on input line 83.
LaTeX Font Info:    Checking defaults for OMX/cmex/m/n on input line 83.
LaTeX Font Info:    ... okay on input line 83.
LaTeX Font Info:    Checking defaults for U/cmr/m/n on input line 83.
LaTeX Font Info:    ... okay on input line 83.
LaTeX Font Info:    Try loading font information for T1+lmr on input line 83.
(/usr/share/texmf/tex/latex/lm/t1lmr.fd
File: t1lmr.fd 2007/01/14 v1.3 Font defs for Latin Modern
)
-------------------- Geometry parameters
paper: letterpaper
landscape: --
twocolumn: --
twoside: --
asymmetric: --
h-parts: 79.49744pt, 455.30013pt, 79.49744pt
v-parts: 72.26999pt, 643.20256pt, 79.49744pt
hmarginratio: --
vmarginratio: --
lines: --
heightrounded: --
bindingoffset: 0.0pt
truedimen: --
includehead: --
includefoot: --
includemp: --
driver: pdftex
-------------------- Page layout dimensions and switches
\paperwidth  614.295pt
\paperheight 794.96999pt
\textwidth  455.30013pt
\textheight 643.20256pt
\oddsidemargin  7.22745pt
\evensidemargin 7.22745pt
\topmargin  -31.8738pt
\headheight 15.2pt
\headsep    19.8738pt
\footskip   30.0pt
\marginparwidth 47.0pt
\marginparsep   7.0pt
\columnsep  10.0pt
\skip\footins  10.8pt plus 4.0pt minus 2.0pt
\hoffset 0.0pt
\voffset 0.0pt
\mag 1000

(1in=72.27pt, 1cm=28.45pt)
-----------------------
LaTeX Font Info:    Try loading font information for T1+lmtt on input line 87.
(/usr/share/texmf/tex/latex/lm/t1lmtt.fd
File: t1lmtt.fd 2007/01/14 v1.3 Font defs for Latin Modern
) [1

{/var/lib/texmf/fonts/map/pdftex/updmap/pdftex.map}]
LaTeX Font Info:    Try loading font information for OT1+lmr on input line 145.

(/usr/share/texmf/tex/latex/lm/ot1lmr.fd
File: ot1lmr.fd 2007/01/14 v1.3 Font defs for Latin Modern
)
LaTeX Font Info:    Try loading font information for OML+lmm on input line 145.

(/usr/share/texmf/tex/latex/lm/omllmm.fd
File: omllmm.fd 2007/01/14 v1.3 Font defs for Latin Modern
)
LaTeX Font Info:    Try loading font information for OMS+lmsy on input line 145
.
(/usr/share/texmf/tex/latex/lm/omslmsy.fd
File: omslmsy.fd 2007/01/14 v1.3 Font defs for Latin Modern
)
LaTeX Font Info:    Try loading font information for OMX+lmex on input line 145
.
(/usr/share/texmf/tex/latex/lm/omxlmex.fd
File: omxlmex.fd 2007/01/14 v1.3 Font defs for Latin Modern
)
LaTeX Font Info:    External font `lmex10' loaded for size
(Font)              <12> on input line 145.
LaTeX Font Info:    External font `lmex10' loaded for size
(Font)              <8> on input line 145.
LaTeX Font Info:    External font `lmex10' loaded for size
(Font)              <6> on input line 145.
[1

] [2

]
LaTeX Font Info:    External font `lmex10' loaded for size
(Font)              <10> on input line 214.
LaTeX Font Info:    External font `lmex10' loaded for size
(Font)              <7> on input line 214.
LaTeX Font Info:    External font `lmex10' loaded for size
(Font)              <5> on input line 214.
[3

] [4] [5] [6] [7] [8] [9] [10] [11] [12

] [13] [14] [1

] [2] [3] [4] [5] [6] [7] [8] (./33663-t.aux)

 *File List*
    book.cls    2005/09/16 v1.4f Standard LaTeX document class
    bk12.clo    2005/09/16 v1.4f Standard LaTeX file (size option)
geometry.sty    2002/07/08 v3.2 Page Geometry
  keyval.sty    1999/03/16 v1.13 key=value parser (DPC)
geometry.cfg
 fontenc.sty
   t1enc.def    2005/09/27 v1.99g Standard LaTeX file
 lmodern.sty    2007/01/14 v1.3 Latin Modern Fonts
fancyhdr.sty    
verbatim.sty    2003/08/22 v1.5q LaTeX2e package for verbatim enhancements
   t1lmr.fd    2007/01/14 v1.3 Font defs for Latin Modern
  t1lmtt.fd    2007/01/14 v1.3 Font defs for Latin Modern
  ot1lmr.fd    2007/01/14 v1.3 Font defs for Latin Modern
  omllmm.fd    2007/01/14 v1.3 Font defs for Latin Modern
 omslmsy.fd    2007/01/14 v1.3 Font defs for Latin Modern
 omxlmex.fd    2007/01/14 v1.3 Font defs for Latin Modern
 ***********

 ) 
Here is how much of TeX's memory you used:
 1137 strings out of 94074
 14633 string characters out of 1165154
 63771 words of memory out of 1500000
 4433 multiletter control sequences out of 10000+50000
 64860 words of font info for 49 fonts, out of 1200000 for 2000
 645 hyphenation exceptions out of 8191
 23i,11n,43p,201b,343s stack positions out of 5000i,500n,6000p,200000b,5000s
{/usr/share/texmf/fonts/enc/dvips/lm/lm-ec.enc}{/usr/share/texmf/fonts/enc/dv
ips/lm/lm-mathex.enc}{/usr/share/texmf/fonts/enc/dvips/lm/lm-mathit.enc}{/usr/s
hare/texmf/fonts/enc/dvips/lm/lm-mathsy.enc}{/usr/share/texmf/fonts/enc/dvips/l
m/lm-rm.enc}</usr/share/texmf/fonts/type1/public/lm/lmbx12.pfb></usr/share/texm
f/fonts/type1/public/lm/lmcsc10.pfb></usr/share/texmf/fonts/type1/public/lm/lme
x10.pfb></usr/share/texmf/fonts/type1/public/lm/lmmi12.pfb></usr/share/texmf/fo
nts/type1/public/lm/lmmi8.pfb></usr/share/texmf/fonts/type1/public/lm/lmr10.pfb
></usr/share/texmf/fonts/type1/public/lm/lmr12.pfb></usr/share/texmf/fonts/type
1/public/lm/lmr17.pfb></usr/share/texmf/fonts/type1/public/lm/lmr7.pfb></usr/sh
are/texmf/fonts/type1/public/lm/lmr8.pfb></usr/share/texmf/fonts/type1/public/l
m/lmri12.pfb></usr/share/texmf/fonts/type1/public/lm/lmsy10.pfb></usr/share/tex
mf/fonts/type1/public/lm/lmsy8.pfb></usr/share/texmf/fonts/type1/public/lm/lmtt
10.pfb></usr/share/texmf/fonts/type1/public/lm/lmtt12.pfb>
Output written on 33663-t.pdf (23 pages, 325386 bytes).
PDF statistics:
 145 PDF objects out of 1000 (max. 8388607)
 0 named destinations out of 1000 (max. 131072)
 1 words of extra memory for PDF output out of 10000 (max. 10000000)

